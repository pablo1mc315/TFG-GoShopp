\thispagestyle{empty}

\noindent{\textbf{Resumen}}\\

El presente proyecto consiste en el desarrollo de una aplicación móvil multiplataforma, es decir, diseñada tanto para el sistema operativo Android como para iOS, y cuyo objetivo es facilitar la tarea de gestionar listas de la compra, tanto personales como entre grupos de conocidos. Con esto, se pretende que cada usuario pueda añadir productos a sus listas privadas o a otros grupos en cualquier momento desde su dispositivo móvil, de forma que cuando uno de los integrantes vaya a hacer la compra, no falte nada. La aplicación también cuenta con una sección de chat, que permite a los usuarios de un grupo poder comunicarse entre ellos a la hora de crear las diferentes listas y unirse al grupo que deseen o crear uno nuevo. Adicionalmente, también se permite a los usuarios escanear el contenido de un ticket de la compra, ya sea mediante la cámara o mediante una imagen obtenida del dispositivo, para así poder añadir productos a una lista de una manera mucho más rápida y sencilla. Esto supone que el propio programa tiene la capacidad de reconocer textos y procesarlos para que, simplemente con una imagen del recibo de la compra, se añadan cada uno de los productos como si se hiciese manualmente. Para desarrollar todo lo mencionado, se ha implementado la aplicación utilizando Flutter, un entorno de trabajo que permite desarrollar simultáneamente aplicaciones tanto para Android como para iOS, además de plataformas externas como Firebase, utilizada en el desarrollo de aplicaciones web y  móviles y, desde hace un tiempo, adquirida por Google.

\vspace{0.7cm}

\noindent{\textbf{Palabras clave}: Flutter, Dart, Firebase, Aplicación móvil, Listas de la compra, Chat, Reconocimiento de textos.}\\

\clearpage

\thispagestyle{empty}

\noindent{\textbf{Abstract}}\\

The present project consists of the development of a multi-platform mobile application, meaning it is designed for both the Android and iOS operating systems. Its objective is to make easier the task of managing shopping lists, both personal and group. Then, each user can add products to their personal lists or to the lists of a group they belong to at any time from their mobile device, so that when one of the members goes shopping, nothing is missing. The application also has a chat section, which allowing group users to communicate with each other when creating the different lists, as well as allowing them to join any group they desire or create a new one. Additionally, users also are able to scan the contents of a purchase receipt, either using the camera or by uploading an image from their device, to add products to a list in a much faster and easier way. This means that the program itself has the ability to recognize texts and process them so that, simply with an image of the purchase receipt, each of the products is added as if it were done manually. To develop all of the aforementioned, the application has been implemented using Flutter, a development environment that allows for the simultaneous development of applications for Android and iOS, and external platforms such as Firebase, used in the development of web and mobile applications, managed by Google.

\vspace{0.7cm}

\noindent{\textbf{Keywords}: Flutter, Dart, Firebase, Cross-platform mobile application, Shopping lists, Chat, Text recognition}\\

\newpage

\noindent\rule[-1ex]{\textwidth}{2pt}\\[4.5ex]

Yo, \textbf{Pablo Muñoz Castro}, alumno de la titulación GRADO EN INGENIERÍA INFORMÁTICA de la \textbf{Escuela Técnica Superior de Ingenierías Informática y de Telecomunicación de la Universidad de Granada}, con DNI 77557076H, autorizo la ubicación de la siguiente copia de mi Trabajo Fin de Grado en la biblioteca del centro para que pueda ser consultada por las personas que lo deseen.

\vspace{2cm}

\noindent Fdo: Pablo Muñoz Castro

\vspace{2cm}

\begin{flushright}
Granada a 4 de julio de 2023.
\end{flushright}

\newpage

\noindent\rule[-1ex]{\textwidth}{2pt}\\[4.5ex]

D. \textbf{Juan José Escobar Pérez}, profesor del Departamento Lenguajes y Sistemas Informáticos de la Universidad de Granada.


\vspace{0.5cm}

\textbf{Informa:}

\vspace{0.5cm}

Que el presente trabajo, titulado \textit{\textbf{Aplicación Móvil para Gestionar la Lista de la Compra}}, ha sido realizado bajo su supervisión por \textbf{Pablo Muñoz Castro}, y autorizo la defensa de dicho trabajo ante el tribunal que corresponda.

\vspace{0.5cm}

Y para que conste, expido y firmo el presente informe en Granada a 4 de julio de 2023.

\vspace{2cm}

\textbf{El tutor:}

\vspace{1cm}

\noindent \textbf{Juan José Escobar Pérez}

\newpage

\section*{Agradecimientos}
\thispagestyle{empty}

       \vspace{1cm}

Quisiera aprovechar este espacio para expresar mi profundo agradecimiento a todas aquellas personas que contribuyeron de manera significativa en la realización de este trabajo de fin de grado. Su apoyo, orientación y aliento fueron fundamentales para culminar este importante logro en mi trayectoria académica:

A mi tutor, Juan José, por su inestimable guía y dedicación a lo largo de estos meses. Agradezco sinceramente su apoyo constante y su capacidad para ofrecer nuevas posibilidades e ideas en el desarrollo de la aplicación.

A todos los profesores que, durante estos cuatro años, se han esforzado en brindar el máximo de sus posibilidades en mi aprendizaje y por proporcionarme las herramientas intelectuales necesarias para desarrollar este proyecto.

A mis compañeros de clase, más bien amigos, quienes me han brindado su apoyo incondicional durante esta etapa. Sus palabras de aliento, su ayuda en estos cuatro años y, principalmente, los momentos de distracción fueron esenciales para mantener mi motivación y confianza en mí mismo, especialmente en una época muy complicada como fue la pandemia.

A mi familia, por su apoyo constante y porque siempre se han esforzado al máximo en todos los aspectos para que nunca me faltase nada, les agradezco su amor, comprensión y paciencia.

A mi novia, por ser mi mayor soporte emocional y por su respaldo constante, que ha sido fundamental para superar los desafíos y obstáculos que surgieron en este camino.

Por ultimo, quiero destacar mi propio esfuerzo y dedicación en este proyecto, el cual se ha convertido tanto en un desafío personal, como en una experiencia de aprendizaje que me ha permitido complementar mis habilidades como desarrollador de software, así como aprender nuevas habilidades y tecnologías.

\newpage
\cleardoublepage